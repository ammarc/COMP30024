\documentclass[twoside]{article}
\usepackage{lmodern}
\title{Artificial Intelligence: Week 1}
\date{}
\author{}
\begin{document}
\maketitle
\section{What is an AI?}
There are various definitions of an AI, ranging from thinking humanly and
rationally to acting humanly and rationally. The \emph{turing test}, is test
in which a human interrogator interacts with a machine, sending it messages
back and forth, and a machine passes if it fools the human into thinking that
the messages are being sent to them by a human. For this a computer needs:
\textbf{natural language processing, knowledge representation, automated
reasoning and machine learning}. To pass the \emph{total turing test} a computer
would additionally need \textbf{computer vision and robotics.}
\section{Intelligent Agents}
An \textbf{agent} is just something that acts and a \textbf{rational agent} is
one that acts so as to achieve the best outcome or, when there is uncertainty,
the best expected outcome. \textbf{Percept} means the agent's perceptual inputs
at any given time, and a \textbf{percept sequence} is the complete history of
everything the agent has ever perceived. The \textbf{agent function} is an
abstract mathematical description that maps a given percept to an action; an
\textbf{agent program} is a concrete implementation of, running within some
physical system. \emph{It is better to design a performance measure according
to what one wants in an environment, then how one wants an agent to behave.}
\\ \\
The proper definition of a rational agent is \emph{for each possible percept
sequence, a rational agent should select an action that is expected to
maximize its performance measure, given the evidence provided by the percept
sequence and whatever built-in knowledge the agent has.} An \textbf{omniscient}
agent knows the actual outcomes of its actions.
\subsection{The Nature of Environments}
An environment is defined as \textbf{PEAS}: performance measure, environment,
actuators and sensors. There are different types of environments, namely:
\begin{itemize}
\item \textbf{Observable vs Partially-Observable}: it is observable when the agents'
        sensors have complete access to the environment's state at all times
\item \textbf{Single agent vs Multi agent}: there could be multiple agents in
        an environment. There is also a question of what must be considered an
        agent. This gives way to the concept of \textbf{competitive} vs
        \textbf{cooperative} environments.
\item \textbf{Deterministic vs Stochastic}: If the next state can be completely
        determined by the current state and the action executed by the agent,
        then it is deterministic; and stochastic otherwise. An environment
        is \textbf{uncertain} if it is not fully observable or not deterministic.
        Note that a \textbf{non-deterministic} environment is one where each
        action is characterized by its possible outcomes, but no probabilities
        are attached to them.
\item \textbf{Episodic vs Sequential}: In an episodic environment the agent's
        experience is divided into atomic episodes. The next episode doesn't
        depend on the action taken in the previous episode.
\item \textbf{Static vs Dynamic}: If an environment can change when an agent
        is deliberating, then it's dynamic, and is static otherwise. If the
        environment doesn't change when deliberating but the performance score
        does, then we call it \textbf{semi-dynamic.}
\item \textbf{Discrete vs Continuous}: The distinction here applies to the state
        of the environment, the way time is handled and the percepts and actions
        of the agent. For example, chess having a discrete set of states; the
        same doesn't apply for taxi driving.
\item \textbf{Known vs Unknown}: This applies to the agent's state of knowledge
        about the ``laws of physics'' of the environment. Note that it's 
        possible that a known environment is partially observable like solitaire.
        Conversely, an environment can also be unknown and fully observable,
        like in a video game, one can see the state but one doesn't know the
        control until one tries to play.
\end{itemize}
\subsection{The Structure of Agents}
\end{document}